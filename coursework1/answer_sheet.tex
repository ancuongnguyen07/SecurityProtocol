\documentclass{article}
\usepackage{graphicx}
\usepackage{amsmath}
\usepackage[parfill]{parskip}
\usepackage{hyperref}
\graphicspath{ {./figures/} }

\title{COMP.SEC.220 Security Protocol\footnote{github --- \url{https://github.com/ancuongnguyen07/SecurityProtocol}}}
\author{Cuong Nguyen --- Coursework 1}
\date{22/04/2022}

\begin{document}
    
\maketitle

\section*{Exercise 1 --- XOR Encryption}
%
Given a message \emph{m} and the OTP encryption \emph{c}, we \textbf{can} compute
the OTP key \emph{k} by XORing the plaintext \emph{m} and the ciphertext \emph{c}.
The ciphertext is given by:

\begin{align*}
    c = m \oplus k,
\end{align*}

therefore, we can easily compute the \emph{k} by:

\begin{align*}
    k = c \oplus m,
\end{align*}

where \emph{c} and \emph{m} are known.

\section*{Exercise 2 --- Substitution Cipher}
%

\section*{Exercise 3 --- Diffie-Hellman}
%
\subsection*{a)}
Alice's intermediate values:

\begin{align*}
    X &= g^{a} \mod p\\
    X &= 7^6 \mod 11\\
    X &= 4 \mod 11
\end{align*}

Bob's intermediate values:

\begin{align*}
    Y &= g^{b} \mod p\\
    Y &= 7^9 \mod 11\\
    Y &= 8 \mod 11
\end{align*}

The final key that Alice and Bob exchange:

\begin{align*}
    K_{AB} &= X^b \mod p\\
    K_{AB} &= 4^9 \mod 11\\
    K_{AB} &= 3 \mod 11
\end{align*}

\subsection*{b)}
Since \(g=7\) is the primitive root of \(p=11\) so we can brute-force
\emph{a} in \(g^a = 5 \mod 11\) in at most 10 trials. Hence, the secret
value of Alice \emph{a} is 2. The secret value of Bob \emph{b}, which is 5, can be
computed by the same approach. Thus, the final key that Alice and Bob exchanged
is given by:

\begin{align*}
    K_{AB} &= X^b = Y^a = g^{ab} \mod p\\
    K_{AB} &= 1 \mod 11 
\end{align*}

\subsection*{c)}
By running the Caesar decryption script with the shift key is 1 (left-shift 1
in this case. In other words, it is right-shift 25), I obtained the plaintext:

\begin{center}
    \textbf{SUEDEJACKETWITHCANDYSTRIPELINING}
\end{center}

\section*{Exercise 4 --- Three party Diffie-Hellman}
%
Waiting for the TA's feedback

\section*{Exercise 5 --- Symmetric Encryption}
%

\end{document}