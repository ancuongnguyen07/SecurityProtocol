\subsection*{a) Step-by-step what the receiver would do}
%
\textbf{Protocol A}

\begin{itemize}
    \item Given \emph{c}, the receiver decrypt with secret key \(k_1\),
    obtaning the following: \(x||H(k_2||x)\).
    \item Hash the combination of \(k_2\) and \emph{x} (\(k_2||x\)),
    obtaining \(H'(k_2||x)\).
    \item Verify if \(H(k_2||x) == H'(k_2||x)\). If they are equal the received message
    is valid. Otherwise, someone intervened the original message. 
\end{itemize}

\textbf{Protocol B}

\begin{itemize}
    \item Given \emph{c}, the receiver decrypt with shared key \emph{k},
    obtaining the following: \(x||\sigma_{pr}(H(x))\).
    \item Decrypting \(\sigma_{pr}(H(x))\) with the public key of the sender,
    obtaning the hash value \(H(x)\).
    \item Hash the received message \emph{x}, obtaining \(H'(x)\)
    \item Verify if \(H(x) == H'(x)\). If they are equal the received message
    is valid. Otherwise, someone intervened the original message. 
\end{itemize}

\subsection*{b) Security properties}
%
\textbf{Protocol A}

\begin{itemize}
    \item \textbf{Confidentiality}: YES as it conducts the encryption with secret
    key \(k_1\).
    \item \textbf{Integrity}: YES as it uses keyed hash function with secret key
    \(k_2\) to verify the integrity.
    \item \textbf{Non-repudiation}: NO as both sender and receiver know the secret
    \(k_2\), one of them can fake the message that it is sent by the other. The
    sender can reject that he/she did send the message
\end{itemize}

\textbf{Protocol B}

\begin{itemize}
    \item \textbf{Confidentiality}: YES as it conducts the encryption with shared
    secret key \emph{k}.
    \item \textbf{Integrity}: YES as it uses the digital signature with the private
    key of the sender to sign and the public key to verify.
    \item \textbf{Non-repudiation}: YES as only the sender know the private key which
    is used to sign the signature. Thus, the sender cannot reject that he/she did not
    send the message.
\end{itemize}