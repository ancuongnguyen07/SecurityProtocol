\subsection*{a)}
%
\textbf{Foward privacy} states that the server cannot realize whether or not a newly added
file contains any keywords used in previous searches.

\textbf{Backward privacy} states that the server cannot detect a previously deleted file
which contains keywords of search queries.

\subsection*{b)}
%
After the following operations, the CSP learn that:
\begin{itemize}
    \item Alice adds \((f_1,w_1)\): adding operation, the size and unique id of \(f_1\)
    \item Alice searches \(f_1\): search pattern, access pattern
    \item Alice adds \((f_2,w_1)\): adding operation, the size and unique id of \(f_2\)
\end{itemize}

\subsection*{c)}
%
The leakage associated with the serach function if:
\begin{itemize}
    \item Backward Privacy Type-I:\ \emph{TimeDB(w)} --- \(f_2\) added at time 2 currently
    stores \(w_1\), 3 updates on \(w\) (2 insertion and 1 deletion).
    \item Backward Privacy Type-II:\ \emph{TimeDB(w),Updates(w)} --- \(f_2\) added at time 2
    currently stores \(w_1\), 3 updates on \(w\) (2 insertion and 1 deletion), updates on
    \(w\) occurs at time 1,2,3.
    \item Backward Privacy Type-III:\ \emph{TimeDB(w),Updates(w),DelHist(w)} --- \(f_2\)
    added at time 2 currently stores \(w_1\), 3 updates on \(w\) (2 insertion and 1 deletion),
    updates on \(w\) occurs at time 1,2,3, and the insertion at time 2 is canceled by the
    deletion at time 3.
\end{itemize}

Some notations:
\begin{itemize}
    \item \emph{TimeDB(w)}: which files currently containing
    \emph{w} and when they were added.
    \item \emph{Updates(w)}: when all updates related to \emph{w} occured.
    \item \emph{DelHist(w)}: which insertion is canceled by
    which deletion.
\end{itemize}